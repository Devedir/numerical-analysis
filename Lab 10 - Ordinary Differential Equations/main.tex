\documentclass{article}
\usepackage[T1]{fontenc}
\usepackage[polish]{babel}
\usepackage[utf8]{inputenc}
\usepackage{graphicx} % Required for inserting images
\usepackage{amsmath}
\usepackage{listings} % code highlighting
\usepackage{xcolor}
\usepackage{float}

\definecolor{codegreen}{rgb}{0,0.6,0}
\definecolor{codegray}{rgb}{0.5,0.5,0.5}
\definecolor{codepurple}{rgb}{0.58,0,0.82}
\definecolor{backcolour}{rgb}{0.95,0.95,0.92}

\lstdefinestyle{mystyle}{
    backgroundcolor=\color{backcolour},
    commentstyle=\color{codegreen},
    keywordstyle=\color{magenta},
    numberstyle=\tiny\color{codegray},
    stringstyle=\color{codepurple},
    basicstyle=\ttfamily\footnotesize,
    breakatwhitespace=false,
    breaklines=true,
    captionpos=b,
    keepspaces=true,
    numbers=left,
    numbersep=5pt,
    showspaces=false,
    showstringspaces=false,
    showtabs=false, 
    tabsize=2
}

\lstset{style=mystyle}
\lstset{ % polish letters in code blocks
  literate={ą}{{\k a}}1
  		     {Ą}{{\k A}}1
           {ż}{{\. z}}1
           {Ż}{{\. Z}}1
           {ź}{{\' z}}1
           {Ź}{{\' Z}}1
           {ć}{{\' c}}1
           {Ć}{{\' C}}1
           {ę}{{\k e}}1
           {Ę}{{\k E}}1
           {ó}{{\' o}}1
           {Ó}{{\' O}}1
           {ń}{{\' n}}1
           {Ń}{{\' N}}1
           {ś}{{\' s}}1
           {Ś}{{\' S}}1
           {ł}{{\l}}1
           {Ł}{{\L}}1
}

\title{MOwNiT - Laboratorium 10:\\
Rozwiązywanie równań różniczkowych zwyczajnych}
\author{Wojciech Dąbek}
\date{21 maja 2024}

\begin{document}

\maketitle

\section{Treści zadań}
\begin{enumerate}
    \item Dane jest równanie różniczkowe (zagadnienie początkowe):
    \[y' + y \cos x = \sin x \cos x \qquad y(0) = 0\]
    Znaleźć rozwiązanie metodą Rungego-Kutty i metodą Eulera.\\
    Porównać otrzymane rozwiązanie z rozwiązaniem dokładnym:
    \[y(x) = e^{-\sin x} + \sin x - 1\]
    \item Dane jest zagadnienie brzegowe:
    \[y'' + y = x \qquad y(0) = 1 \qquad y\left(\frac{\pi}{2}\right) = \frac{\pi}{2} - 1\]
    Znaleźć rozwiązanie metodą strzałów.\\
    Porównać otrzymane rozwiązanie z rozwiązaniem dokładnym:
    \[y(x) = \cos x - \sin x + x\]
\end{enumerate}

\newpage

\section{Rozwiązania}

\subsection{Zadanie 1 - zagadnienie początkowe}

Metoda Rungego-Kutty opiera się na wzorze rekurencyjnym:
\[y_{k+1} = y_k + \frac{h}{6}(k_1 + 2k_2 + 2k_3 + k_4)\]
gdzie \(h = y_{k+1} - y_k\) oraz
\begin{gather*}
    k_1 = f(x_k, y_k)\\
    k_2 = f(x_k + \frac{h}{2}, y_k + \frac{h}{2} k_1)\\
    k_3 = f(x_k + \frac{h}{2}, y_k + \frac{h}{2} k_2)\\
    k_3 = f(x_k + h, y_k + h k_3)\\
    f(x, y) = y'
\end{gather*}

\noindent
Metoda Eulera opiera się na prostszym wzorze:
\[y_{k+1} = y_k + h f(x_k, y_k)\]

\noindent
Obliczenia zrealizowałem następującym kodem w języku Python:
\begin{lstlisting}[language=Python]
from math import exp, sin, cos, fabs


def exact(x: float) -> float:
    return exp(-sin(x)) + sin(x) - 1.

def fun(x: float, y: float) -> float:
    return sin(x) * cos(x) - y * cos(x)

def runge_kutta(x_0, y_0, n, h, f):
    x = x_0
    y = y_0
    for _ in range(n):
        k1 = f(x, y)
        k2 = f(x + h/2, y + k1 * h/2)
        k3 = f(x + h/2, y + k2 * h/2)
        k4 = f(x + h,   y + k3 * h)
        y += (k1 + 2*k2 + 2*k3 + k4) * h/6
        x += h
    return y

def euler(x_0, y_0, n, h, f):
    x = x_0
    y = y_0
    for _ in range(n):
        y += h * f(x, y)
        x += h
    return y

if __name__ == '__main__':
    for k in range(1, 6):
        n = 10**k
        print(f'For {n = }:')
        rk_method = runge_kutta(0., 0., n, 1./ n, fun)
        e_method = euler(0., 0., n, 1. / n, fun)
        expect = exact(1.)
        print(f'    Expected value: {expect}')
        print(f'Runge-Kutta method: {rk_method}', end='\t\t')
        print(f'Difference: {fabs(rk_method - expect):.5e}')
        print(f"    Euler's method: {e_method}", end='\t\t')
        print(f'Difference: {fabs(e_method - expect):.5e}')
        print()
\end{lstlisting}

\noindent
Jako wartość \(h\) przyjąłem \(\frac{1}{n}\), aby \(x_n = 1\). Rezultatem działania tego programu jest:

\begin{verbatim}
For n = 10:
    Expected value: 0.27254693545348885
Runge-Kutta method: 0.2725471473929363    Difference: 2.11939e-07
    Euler's method: 0.26442725830740726	  Difference: 8.11968e-03

For n = 100:
    Expected value: 0.27254693545348885
Runge-Kutta method: 0.2725469354731914    Difference: 1.97026e-11
    Euler's method: 0.2718062028296542    Difference: 7.40733e-04

For n = 1000:
    Expected value: 0.27254693545348885
Runge-Kutta method: 0.27254693545349107	  Difference: 2.22045e-15
    Euler's method: 0.2724735390106294    Difference: 7.33964e-05

For n = 10000:
    Expected value: 0.27254693545348885
Runge-Kutta method: 0.27254693545348246	  Difference: 6.38378e-15
    Euler's method: 0.27253960254408427	  Difference: 7.33291e-06

For n = 100000:
    Expected value: 0.27254693545348885
Runge-Kutta method: 0.27254693545351943	  Difference: 3.05866e-14
    Euler's method: 0.2725462022298938    Difference: 7.33224e-07
\end{verbatim}

\vspace{5mm}
\noindent
\textbf{Wnioski:} Jak widać, metoda Rungego-Kutty daje wyraźnie większą dokładność od prostszej metody Eulera. Rośnie ona wraz ze wzrostem \(n\) aż do pewnej dużej dokładności (co może wynikać z dokładności reprezentacji zmiennoprzecinkowej liczb). W przypadku metody Eulera przy eksponencjolnym wzroście \(n\) dokładność wyników rośnie liniowo.

\newpage

\subsection{Zadanie 2 - zagadnienie brzegowe}

\section{Bibliografia}
Prof. M. T. Heath - \textit{Scientific Computing: An Introductory Survey}, ch. 9, 10

\end{document}
